\chapter*{Appendix I: Data Sets and functions}

\section{Connect-Four}
\label{sec:uci-c4}
\mydefs{UCI Connect-Four data set} {An data set obtained from the UCI machine learning repository \cite{bache:uci}. It contains the \myterms{legal position} at the 8-\myterm{ply} in the game of \myterm{Connect-Four} in which neither player has won and the next move is not forced.}

This data set, $\myset D$ has $67557$ examples and $42$ attributes.  Each example is divided into $3$ classes:
\begin{enumerate}
	\item south player wins (i.e. Win class), $s$, has $44473$ elements (about 66\%)
	\item north player wins (i.e. Loss class), $n$, has $16635$ elements (about 24\%)
	\item draws, $d$, has $6449$ elements (about 10\%)
\end{enumerate}

\mydefs{ADATE function} {A function for \myterm{Connect-Four} \myterms{position} derived by Stenmark's experiment \cite{stenmark:masters}. The function is calculated as follows:
\[ 
m = \sum_{i \in S}{p(s) \times w(s)}
\] 
where $p(s) = 1$ if a square, $s$ is occupied by the player, $-1$ when occupied by the opponent and $0$ when it is not occupied.  Also, $w(s)$ is the weight of square $s$. Weights are assigned according to the values in the table below. 
\newline
\begin{center}
\begin{tabular} {|l|l|l|l|l|l|l|}
	\hline
	2& 0&2& 2& 2& 2&1 \\ \hline
	0& 2&6& 6& 2& 4&1 \\ \hline
	0&12&6&14&12&11&2 \\  \hline
	0& 1&4&16& 0& 0&2 \\ \hline
	0& 2&5& 0& 5& 4&4 \\ \hline
	0& 2&0& 1& 12&0&0 \\ \hline
\end{tabular}
\end{center}
} 

\mydefs{IBEF function} {An evaluation function for \myterm{Connect-Four} \myterms{position} defined by Stenmark \cite{stenmark:masters}.  It is an acronym for  {\it Intuitive Board Evaluation Function}.  It uses the same calculation method as the \myterm{ADATE function} but employs different weights, as given below 
\newline
\begin{center}
\begin{tabular} {|l|l|l|l|l|l|l|}
	\hline
	3&4& 5& 7& 5&4&3 \\ \hline
	4&6& 8&10& 8&6&4 \\ \hline
	5&8&11&13&11&8&5 \\ \hline
	5&8&11&13&11&8&5 \\ \hline
	4&6& 8&10& 8&6&4 \\ \hline
	3&4& 5& 7& 5&4&3 \\ \hline
\end{tabular}
\end{center}
}. 

\chapter*{Appendix II: Mathematics}
\section{Double sum equality}
\label{sec:sum_proof}
We need to prove that:
\begin{align}
\sum_{i=1}^n\sum_{j=i}^n a_ib_j = \sum_{j=1}^n\sum_{i=1}^j a_ib_j
\end{align}

For $n = 1$:
\begin{align}
\sum_{i=1}^1\sum_{j=i}^1 a_ib_j &= a_1b_1 \\
\sum_{j=1}^1\sum_{i=1}^j a_ib_j &= a_1b_1 
\end{align}

Assuming equality holds for n, for $n+1$:
\begin{align}
\sum_{i=1}^{n+1}\sum_{j=i}^{n+1} a_ib_j 
&= \sum_{i=1}^{n+1}\left(\sum_{j=i}^{n} a_ib_j + a_ib_{n+1}\right) \\
&= \sum_{i=1}^{n}\left(\sum_{j=i}^{n} a_ib_j + a_ib_{n+1}\right) 
+ \left(\sum_{j=n+1}^{n} a_{n+1}b_j + a_{n+1}b_{n+1}\right) \\
&= \sum_{i=1}^{n}\left(\sum_{j=i}^{n} a_ib_j + a_ib_{n+1}\right) 
+ a_{n+1}b_{n+1} \\
&= \sum_{i=1}^{n}\sum_{j=i}^{n} a_ib_j + \sum_{i=1}^{n} a_ib_{n+1}
+ a_{n+1}b_{n+1} \\
&= \sum_{j=1}^n\sum_{i=1}^j a_ib_j + \sum_{i=1}^{n} a_ib_{n+1}
+ a_{n+1}b_{n+1} \\
&= \sum_{j=1}^n\sum_{i=1}^j a_ib_j + \sum_{i=1}^{n+1} a_ib_{n+1} \\
&= \sum_{j=1}^{n+1}\sum_{i=1}^j a_ib_j 
\end{align}